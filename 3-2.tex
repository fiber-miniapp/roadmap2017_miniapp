\section{ミニアプリとの対応}
\label{sec:ミニアプリ}

\subsection{はじめに}

計算機ハードウェア・ソフトウェアの性能や機能を評価するためにベンチマークソフトウェアが用いられているが、それらは典型的にはアプリケーションから限られた一部を切り出したプログラムであったり、人工的に作成されたプログラムである。カーネルベンチマークと呼ばれるこれらの評価用プログラムは、通常のアプリケーションと比較してそのサイズは大幅に小さいものであり、システムの評価に用いやすい利点がある。一方で、カーネルベンチマークは実際のアプリケーションのごく一部のみを反映したものであり、実際のアプリケーション全体を評価できているとは限らない。しかしながら、特に開発途中のシステムの評価に巨大な実アプリケーションを用いることは技術的に困難であり、カーネルベンチマークのような簡易なプログラムが必要である。

より実際のアプリケーションに即しつつ、プログラムサイズを比較的小さなものとし評価に利用し易くすることを目的としたベンチマークとして、ミニアプリベンチマークが提唱されている。ミニアプリは実アプリケーションから評価に必要な箇所を残し、それ以外の非本質的なコードを可能な限り削除することでプログラム全体の見通しを改善する。また、コンパイル方法や実行方法などの文書化、入力データの整備などがあわせて必要である。実アプリケーションには限定的なライセンスにより配布が制限されているものも多いが、そのような制約は限定された範囲内における利用で問題とならないかもしれないが、第三者によるベンチマークとしての利用の場合は利用に関する制約は大きな障害となりうる。そのためミニアプリが非公開アプリケーションから作成された場合であってもミニアプリについてはオープンソースとすることが一般的である。

上記のような背景のもとに作成されたミニアプリは特にアプリケーションと計算機システムのコデザインに有効なツールとして用いられている。特に代表的なものとしては米国のコデザインセンターを中心に開発されたMantevoやLuleshなどがあげられる。国内では理化学研究所が中心となって整備したFiberがあげられる。以下ではそれらについて概要を紹介し、本ロードマップのアプリケーションとの比較を示す。

%\subsection{Fiberミニアプリ}
\input 3-2-2.tex

%\subsection{その他のミニアプリ関連プロジェクト}
%\subsection{2章アプリとミニアプリの対応}
\input 3-2-3.tex

% 参考文献
\nocite{*}
\bibliographystyle{\rmbibstyle}
\bibliography{3-2}

