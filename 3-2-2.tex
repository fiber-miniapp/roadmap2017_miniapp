\subsection{Fiberミニアプリ}
Fiberミニアプリ集(Fiber Miniapp Suite)~\cite{Fiber}は、理化学研究所および東京工業大学にて実施した「将来のHPCIシステムのあり方の調査研究「アプリケーション分野」」~\cite{ApliFS}(以下、アプリFSと呼ぶ)において整備開発したミニアプリ群をまとめたものである。
%アプリFSでは、2018〜2020年頃の社会的・科学的課題を担うアプリケーションの調査を国内の数多くの計算科学者の協力の下、2012年度から2013年度にかけて実施され、その調査結果は「計算科学ロードマップ(2014年3月版)」としてまとめられた~\cite{Roadmap_2014}。
アプリFS参加者から提供化されたアプリケーションプログラムをベースに、
その開発者の協力のもとで、完成後の一般公開を前提に、ミニアプリの整備開発が行われた。
最終的には、
著作権的に問題のあったもの、性能面で問題のあったもの、オリジナル開発者からのサポートが期待できなくなったおものを除いた、8つのミニアプリを現在Web上で公開している。

Fiberでは、前節で示した一般的なミニアプリの共通点の他に、
以下のような特徴を持ったミニアプリ集を目指して開発整備を行った。

\begin{itemize}
\item 容易にインストール、容易に実行
\item 正しくインストールされ、正しく実行されていることが検証できるテスト
\item 性能計測に使える入力データセット
\item インストール・実行方法、使用アルゴリズム説明だけでなく、将来(2018〜2020年頃)の想定問題規模とそこでの目標性能に関する記述を含めたドキュメントの整備
 \end{itemize}

また、その開発経緯により、ミニアプリ化されたアプリケーションは全て国内で開発されたものであり、
また、京コンピュータなど国内固有のアーキテクチャ上でのインストール・実行が保証されているものとなっている。

以下では、Fiberミニアプリ集として公開中の各ミニアプリに対して、
次の点を中心に簡単に紹介する。
\begin{itemize}
\item 計算対象と計算手法
\item 使用言語と並列化ライブラリ
\item どのようにミニアプリ化したか
\item ソースコードサイズ       
\item 付随する入力データ
\end{itemize}

\rmproject{CCS QCD}
ミニアプリCCS QCD~\cite{CCS-QCD_boku2012}\cite{CCS-QCD_terai2013}は、
高エネルギー物理学で用いられる格子量子色力学(格子QCD)計算における、
最も計算コストがかかるクォーク伝搬関数の計算部部分を抜き出したものである。
CCS QCDでは、Wilson型の作用を用いたクォーク伝搬関数の4次元格子上での大規模疎行列連立1次方程式を、
red/black orderingにより前処理をした係数行列に対して、BiCGStab法により解いている。
プログラムはFortran90で記述され,並列化は,空間3次元に対するMPI領域分割とOpenMPスレッド並列を行っている。
コメント行を除いたソースコードサイズは約1000行である。
パッケージには、強スケーリングおよび弱スケーリングでの性能計測に対応可能な6つの入力データセットが含まれている。

\rmproject{FFVC Mini}
ミニアプリFFVC Miniは,熱流体解析プログラムFFV-C~\cite{FFVC_ono2014}をベースとしている。
Navier-Stokes方程式を直交等間隔格子上の有限体積法により離散化し。
非圧縮流体として、時間ステップ毎に圧力Poisson方程式を反復法により解いている。
プログラム全体の制御はC++で記述されているが、ホットスポット部分はFortran90により書かれている。
並列化は、空間3次元に対するMPI領域分割とOpenMPスレッド並列を行っている。
ミニアプリ化にともない、計算内容を3次元直方体領域内のキャビティフロー問題に特化し、
使用アルゴリズムも、時間積分は1次精度Euler陽解法、対流項は3次精度MUSCLスキーム、
反復解法はストライドメモリアクセス型の2色SOR法に限定している。
コメント行を除いたソースコードサイズは約9000行である。
ミニアプリ内で、強スケーリングおよび弱スケーリングでの性能計測に対応可能なグリッドデータを自動生成する。

\rmproject{NICAM-DC Mini}
ミニアプリNICAM-DC Miniは、
全地球規模での気象現象をシミュレーションする大気大循環モデルのアプリケーションNICAM~\cite{NICAM_satoh2008}のサブセットNICAM-DC~\cite{NICAM-DC_url}をベースにしている。
NICAM-DCは、NICAMから、より要求B/F値が高く、通信回数の8割が集中している力学過程(Dinamical Core)のみを抜き出したミニアプリとなっている。
NICAM-DCでは、地球大気の運動を静水圧近似を行わないNavier-Stokes方程式で記述し、
それを球殻上の三次元格子を用いて有限体積法により離散化して解いている。
水平方向の格子は、正二十面体を構成する正三角形要素を再帰的に分割していくことにより得られる全球で一様な三角格子を採用している。
時間積分は。水平方向はRunge-Kutta法により陽解法で、鉛直方向は陰解法として1次元Helmholtz方程式を離散化した三重対角行列を係数に持つ連立一次方程式を解くことにより進めている。
NICAM-DCは、Fortran90で記述され、MPIによる領域分割並列化がなされている。
コメント行を除いたソースコードサイズはは3万5千行である。
パッケージには、強スケーリングおよび弱スケーリングでの性能計測に対応可能な5つの入力データセットが付随している。

\rmproject{mVMC Mini}
ミニアプリmVMC Miniは、強相関電子系シミュレーションプログラムmVMC~\cite{mVMC_url}\cite{mVMC_tahara2008}をベースとしている。
mVMCは,近藤格子などの多体量子系有効模型の基底状態の波動関数を多変数変分モンテカルロ法により求める。
mVMCは、C言語で記述され、MPIおよびOpenMPにより並列化されている。
コメント行を除いたソースコードサイズは約9000行である。
パッケージには、強スケーリングおよび弱スケーリングでの性能計測に対応可能な2つの入力データセットが付随している。

\rmproject{NGS Analyzer Mini}
ミニアプリNGS Analyzer MiniのベースとなったNGS Analyzerは、
次世代シークエンサーの出力データを高速に解析し、
ヒト個人間の遺伝的差異やがんゲノムの突然変異を高い精度で同定するプログラムである~\cite{NGSA_url}。
(to be written)

\rmproject{MODYLAS Mini}
ミニアプリMODYLAS Miniは、古典分子動力学シミュレーションプログラムMODYLAS~\cite{Modylas_url}\cite{Modylas_andoh2013}をベースにしている。
MODYLASでは、クーロン相互作用を、八分木構造を持つ空間セル上での多重極展開計算により計算するFast Multipole Method (FMM)法を採用している。
MODYLASはFortran90で記述され、MPIおよびOpenMPによる並列化がなされている。
ミニアプリ化にともない、計算対象を水分子系のミクロカノニカル計算に限定している。
コメント行を除いたソースコードサイズは8千行である。
パッケージには、強スケーリングおよび弱スケーリングでの性能計測に対応可能な3つの入力データが付随している。

\rmproject{NTChem Mini}
ミニアプリNTChem Miniは、第一原理計算にもとづく電子状態計算プログラムNTChem~\cite{NTChem_url}のサブパッケージNTChem/RI-MP2~\cite{NTChem_katouda2013}をベースにしている。
演算は密行列に対する行列行列積演算が中心となる。
NTChem/RI-MP2では、XXXXとして計算している。
NTChem Miniは、Fortran90で記述され、MPIおよびOpenMPにより並列化されている。
コメント行を除いたソースコードサイズは約XXX行である。
パッケージには小規模分子系のサンプル入力データのみが付属するが、
別途、強スケーリング性能計測に対応可能な入力データをWebページから入手可能である。

\rmproject{FFB Mini}
ミニアプリFFB Miniは、有限要素法による熱流体解析プログラムFrontFlow/blue(FFB)~\cite{FFB_minami2012}\cite{FFB_kumahata2013}をベースにしている。
ミニアプリ化にあたり、計算対象を六面体要素による非定常流体解析計算に限定し、
その他にもオリジナルFFBに含まれる多くの機能(乱流モデル、キャビテーションモデル、など)を削除した。
FFB MiniはFortran90で記述され、MPIにより領域分割並列化に対応している。
また、京コンピュータおよび富士通FX10上での、自動並列によるスレッド並列計算に対応している。
コメント行を除いたソースコードサイズは約XXX行である。
ミニアプリ内で、強スケーリングおよび弱スケーリングでの性能計測に対応可能なグリッドデータを自動生成する。

