\documentclass[11pt,a4paper]{jsbook}
\usepackage{lineno}
\usepackage{here}
\usepackage{chapterbib}
\usepackage[dvipdfmx]{graphicx,color,hyperref}
\usepackage{pxjahyper}
\hypersetup{%
  pdftitle={計算科学ロードマップ},
  pdfauthor={今後のHPCIを使った計算科学発展のための検討会},
  bookmarksnumbered=true,
  colorlinks=true,
  citecolor=black,
  filecolor=black,
  linkcolor=black,
  urlcolor=black,
}

\usepackage{color}
\usepackage{colortbl}
\usepackage{array}
\usepackage{booktabs}
\usepackage{longtable}
\usepackage{threeparttablex}
\usepackage{wrapfig}

\pagewiselinenumbers 

%%% Added by YAMAZAKI Tadashi, Editor for Brain Science and AI
\usepackage{otf}
\usepackage{listings,jlisting}
\usepackage{amsmath,amssymb}
\usepackage{enumerate}
\lstset{% 
language={C}, 
backgroundcolor={\color[gray]{.85}},% 
basicstyle={\small\ttfamily},% 
identifierstyle={\small\ttfamily},% 
commentstyle={\small\ttfamily \color[rgb]{0.5,0.5,0.5}},% 
keywordstyle={\small\ttfamily \color[rgb]{0,0,0}},% 
ndkeywordstyle={\small\ttfamily},% 
stringstyle={\small\ttfamily}, 
frame={tb}, 
breaklines=true, 
columns=[l]{fullflexible},% 
%numbers=left,% 
numbers=none,% 
xrightmargin=0zw,% 
xleftmargin=3zw,% 
numberstyle={\scriptsize},% 
stepnumber=1, 
numbersep=1zw,% 
escapechar={\^},%
morecomment=[l]{//}% 
} 
%%%

\input macro.tex

\setlength{\textwidth}{\fullwidth}
\setlength{\evensidemargin}{\oddsidemargin}

% 参考文献スタイル
%\newcommand{\rmbibstyle}{jplain}
\newcommand{\rmbibstyle}{junsrt}

\begin{document}
\title{計算科学ロードマップ\\
(平成28年度版)}
\author{今後のHPCIを使った計算科学発展のための検討会}
\date{平成29年3月発行予定}
\maketitle

\tableofcontents

\mainmatter

%\chapter{序論}
%\include{1-1} % はじめに
%\include{1-2} % 分野連携・大規模実験施設
%\include{1-3} % 将来実現しうる大規模計算機

%\chapter{各計算科学分野の課題}
%\include{2-1} % 素粒子・原子核
%\include{2-2} % ナノサイエンス・デバイス
%\include{2-3} % エネルギー・材料
%\include{2-4} % 生命科学
%\include{2-5} % 創薬・医療
%\include{2-6} % 設計・製造
%\include{2-7} % 社会科学
%\include{2-8} % 脳科学・人工知能
%\include{2-9} % 地震・津波
%\include{2-10} % 気象・気候
%\include{2-11} % 宇宙・天文

\setcounter{chapter}{2}
\chapter{アプリケーションの分類}
\section{計算機アーキテクチャから見たアプリケーションの分類}
\label{sec:計算機アーキテクチャから見たアプリケーションの分類}
 % 計算機アーキテクチャから見たアプリケーションの分類
\section{ミニアプリとの対応}
\label{sec:ミニアプリ}

\subsection{はじめに}

%\subsection{Fiberミニアプリ}
\input 3-2-2.tex

%\subsection{その他のミニアプリ関連プロジェクト}
%\subsection{2章アプリとミニアプリの対応}
\input 3-2-3.tex

% 参考文献
\nocite{*}
\bibliographystyle{\rmbibstyle}
\bibliography{3-2}

 % ミニアプリとの対応

%\chapter{各課題の詳細}
%\include{4-1} % 素粒子・原子核
%\include{4-2} % ナノサイエンス・デバイス
%\include{4-3} % エネルギー・材料
%\include{4-4} % 生命科学
%\include{4-5} % 創薬・医療
%\include{4-6} % 設計・製造
%\include{4-7} % 社会科学
%\include{4-8} % 脳科学・人工知能
%\include{4-9} % 地震・津波
%\include{4-10} % 気象・気候
%\include{4-11} % 宇宙・天文

%\chapter{おわりに}
%\input 5.tex

%\appendix
%\chapter{用語集}
%\input glossary.tex
%
%\chapter{執筆者一覧}
%\input authors.tex

\end{document}
